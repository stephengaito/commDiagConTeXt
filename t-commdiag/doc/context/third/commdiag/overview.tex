% A ConTeXt document [master document: commdiag.tex]

\startchapter[title=Overview]

This document defines the Commutative Diagrams for \ConTeXt\ module, 
\type{t-commdiag}. This module provides (Mathematical) Commutative 
Diagrams using \ConTeXt's underlying MetaPost/MetaFun graphics primitives. 

This document uses the \type{t-literate-modules} \ConTeXt\ module to 
define the \type{t-commdiag} module. This means that the code required to 
implement the Commutative Diagrams module is an integral part of this document. 

As far as the \type{t-commdiag} module is concerned, a commutative diagram 
is a collection of \quote{objects} laid out in a (rectangular) martix, 
together with a collection of \quote{arrows} between pairs of 
\quote{objects}. The label for a given object is typeset using \ConTeXt\ 
and is then drawn in the appropriate location in the matrix. Each arrow 
can be drawn with a number of arrow bodies, as well as arrow heads and/or 
tails. Each arrow can also have a label, again typeset in \ConTeXt. 

To \emph{use} the \type{t-commdiag} module, you need to \type{use} the 
\type{t-commdiag} module in the \quote{usual} \ConTeXt\ way: 

\starttyping
\usemodule[t-commdiag]
\stoptyping

\noindent\ This \type{\usemodule} statement must occur in the document's 
setup section.

\startbuffer
\startformula \startMPcode{commDiag}  
  setupCommDiags ; 
  
  addObject(1,1, "X_0");
  addObject(1,2, "X_1");
  
  addObject(2,1, "Y_0");
  addObject(2,2, "Y_1");

  drawObjects(1.75cm, 2cm);

  addArrow(1,1, 1,2, ">", 0)()()("j_0",     0.5, top);
  addArrow(1,1, 2,1, ">", 0)()()("k_0",     0.5, lft);
  addArrow(1,2, 2,2, ">", 0)()()("k_1",     0.5, rt);
  addArrow(2,1, 2,2, ">", 0)()()("t_1",     0.5, bot);
\stopMPcode \stopformula
\stopbuffer

As a simple example, consider the following square commutative diagram: 
\processTEXbuffer

\noindent\ It was created using the following \type{t-commdiag} code:

\typebuffer

A typical \type{t-commdiag} diagram's code is enclosed in a \ConTeXt\ 
MetaFun environment (in this case the pair of \type{\startMPcode} and 
\type{\stopMPcode} \ConTeXt\ macros). While a \type{t-commdiag} 
commutative diagram \emph{can} be position inline (inside running text), 
it is typically positioned using one of \ConTeXt's displayed mathematics 
or figure environments (in this case the pair of \type{\startformula} and 
\type{\stopformula} \ConTeXt\ macros). 

The \type{t-commdiag} code which describes a given commutative diagram has 
five sections: 

\startitemize[n]

\item {\bf The \type{setupComDiags} statement.} The \type{setupCommDiags} 
statement sets up the \type{objLabel} (object labels), and \type{objPos} 
(object position) matrices, as well as the \type{numRows} and 
\type{numCols} variables. These \type{objLabel} and \type{objPos} matrices 
are used by the \type{addObject} and \type{addArrow} statements, to 
correctly align the objects and arrows of the commutative diagram. They 
can also be used in your own MetaPost/MetaFun code. 

\item {\bf A collection of \type{addObject} statements.} The 
\type{addObject} macro takes three arguments. The first two are the row 
and column in the overall commutative diagram's matrix in which the object 
is located. The third and last argument is the object label as a string, 
which is typeset by \ConTeXt. The row and column are expected to be 
integers and are indices into the \type{objLabel} and \type{objPos} 
matrices. 

\item {\bf The \type{drawObjects} statement.} The two arguments to the 
\type{drawObjects} macro, are the row height, and column width 
respectively. You can use these two parameters to adjust the overall size 
of your diagram. The \type{drawObjects} statement sets the locations of 
all of the object labels and then draws them. You can use these object 
labels or object positions in your own MetaPost/MetaFun code. 

\item {\bf A collection of \type{addArrow} statements.} The 
\type{addArrow} macro is rather more complex and is discussed below. 

\item {\bf Optionally, your own MetaPost/MetaFun code.} If for some reason 
the \type{t-commdiag} primitives are not sufficient for your needs, you 
are able to use the \type{objLabel} and \type{objPos} matrices in your 
arbitrary MetaPost/MetaFun code. However your own code \emph{must} be 
located \emph{after} the \type{drawObjects} statement, and preferably 
after all \type{addArrow} statements. 

\stopitemize

\noindent\ The order of these four sections is \emph{critical}, they will 
not work in any other order.

\stopchapter 