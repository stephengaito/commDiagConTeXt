% A ConTeXt document [master document: commdiag.tex]

\startchapter[title=Overview]

This document defines the Commutative Diagrams for \ConTeXt\ module, 
\type{t-commdiag} \ConTeXt\ module. This module provides (Mathematical) 
Commutative Diagrams using \ConTeXt's underlying MetaPost/MetaFun graphics 
primitives. 

This document uses the \type{t-literate-modules} \ConTeXt\ module to 
define the \type{t-commdiag} module. This means that the code required to 
implement the Commutative Diagrams module is an integral part of this document. 

As far as the \type{t-commdiag} module is concerned, a commutative diagram 
is a collection of \quote{objects} laid out in a (rectangular) martix, 
together with a collection of \quote{arrows} between pairs of 
\quote{objects}. The label for a given object is typeset using \ConTeXt\ 
and is then drawn in the appropriate location in the matrix. Each arrow 
can be drawn with a number of arrow bodies, as well as arrow heads and/or 
tails. Each arrow can also have a label, again typeset in \ConTeXt. 

\section[title=Using \type{t-commdiag}]

To \emph{use} the \type{t-commdiag} module, you need to \type{use} the 
\type{t-commdiag} module in the \quote{usual} \ConTeXt\ way: 

\starttyping
\usemodule[t-commdiag]
\stoptyping

\noindent\ This \type{\usemodule} statement must occur in the document's 
setup section.

\section[title=Describing a commutative diagram]

\startbuffer
\startformula \startMPcode{commDiag}  
  setupCommDiags ; 
  
  addObject(1,1, "X_0");
  addObject(1,2, "X_1");
  
  addObject(2,1, "Y_0");
  addObject(2,2, "Y_1");

  drawObjects(1.75cm, 2cm);

  addArrow(1,1, 1,2, ">", 0)()()("j_0",     0.5, top);
  addArrow(1,1, 2,1, ">", 0)()()("k_0",     0.5, lft);
  addArrow(1,2, 2,2, ">", 0)()()("k_1",     0.5, rt);
  addArrow(2,1, 2,2, ">", 0)()()("t_1",     0.5, bot);
\stopMPcode \stopformula
\stopbuffer

As a simple example, consider the following square commutative diagram: 
\processTEXbuffer

\noindent\ It was created using the following \type{t-commdiag} code:

\typebuffer

A typical \type{t-commdiag} diagram's code is enclosed in a \ConTeXt\ 
MetaFun environment (in this case the pair of 
\type{\startMPcode{commDiag}}\footnote{The \ConTeXt\ MetaFun macros all 
work within named MetaPost library instances. This ensures differnt uses 
of MetaPost are properly isolated. The \type{t-commdiag} module follows 
this practice by defining all of its MetaPost/MetaFun macros inside the 
\type{commDiag} instance. This means that the \type{commDiag} argument to 
the \ConTeXt\ MetaPost/MetaFun environments is \emph{required}.} and 
\type{\stopMPcode} \ConTeXt\ macros). While a \type{t-commdiag} 
commutative diagram \emph{can} be position inline (inside running text), 
it is typically positioned using one of \ConTeXt's displayed mathematics 
or figure environments (in this case the pair of \type{\startformula} and 
\type{\stopformula} \ConTeXt\ macros). 

The \type{t-commdiag} code which describes a given commutative diagram has 
five sections: 

\startitemize[n]

\item {\bf The \type{setupComDiags} statement} sets up the \type{objLabel} 
(object labels), and \type{objPos} (object position) matrices, as well as 
the \type{numRows} and \type{numCols} variables. These \type{objLabel} and 
\type{objPos} matrices are used by the \type{addObject} and 
\type{addArrow} statements, to correctly align the objects and arrows of 
the commutative diagram. These matrices can also be used in your own 
MetaPost/MetaFun code. 

\item {\bf A collection of \type{addObject} statements.} The 
\type{addObject} macro takes upto four arguments. The first two are the 
row and column in the overall commutative diagram's matrix in which the 
object is located. The third argument is the object label as a string, 
which is typeset by \ConTeXt. The optional last argument, in standard 
MetaPost style, is specified by the text \emph{after} the closing 
\type{")"} and the statement end, \type{";"}. Any picture modification 
macro can be used in this optional last argument. The row and column 
arguments are expected to be integers and are indices into the 
\type{objLabel} and \type{objPos} matrices. You are not required to supply 
an object for every row, column combination. Matrix cells which have no 
object label are simply not drawn\footnote{The object position values 
\emph{are} computed should they be needed in your own code.}. 

\item {\bf The \type{drawObjects} statement.} The two arguments to the 
\type{drawObjects} macro, are the row height, and column width 
respectively. You can use these two parameters to adjust the overall size 
of your diagram. The \type{drawObjects} statement sets the locations of 
all of the object labels and then draws them\footnote{MetaPost has the 
native ability to compute layouts given a collection of drawing 
constraints. However, at this time, we keep things simple by 
\emph{explicitly} supplying the row height and column width. At sometime 
in the future, we may implement full \emph{implicit} layout using 
MetaPost's native constraint solving capabilities.}. You can use these 
object labels or object positions in your own MetaPost/MetaFun code. 

\item {\bf A collection of \type{addArrow} statements.} The 
\type{addArrow} macro is rather more complex and is discussed below. 

\item {\bf Optionally, your own MetaPost/MetaFun code.} If for some reason 
the \type{t-commdiag} primitives are not sufficient for your needs, you 
are able to use the \type{objLabel} and \type{objPos} matrices in your own 
MetaPost/MetaFun code. However your own code \emph{must} be located 
\emph{after} the \type{drawObjects} statement, and preferably after all 
\type{addArrow} statements. 

\stopitemize

\noindent\ The order of these five sections is \emph{critical}, they will 
not work in any other order. 

\section[title=The \type{addArrow} macro]

{\bf The \type{addArrow} macro} has upto 12 arguments in a number of 
logical groups. The {\bf arrow is described} by the first six arguments. 
The next two arguments, as well as the last (optional) argument 12, all 
provide you the ability to specify {\bf MetaPost/MetaFun picture 
commands}. Arguments 9, 10 and 11 describe the arrow's label. 

\startitemize[n]

\item {\bf Arrow description } \type{fromRow} integer index. 

\item {\bf Arrow description } \type{fromCol} integer index. 

\item {\bf Arrow description } \type{toRow} integer index. 

\item {\bf Arrow description } \type{toCol} integer index. Together these 
four indices, \type{(fromRow, fromCol)} and \type{(toRow, toCol)}, specify 
the objects which the arrow connects. 

\item {\bf Arrow description } \type{arrowDecorations} is a string which 
describes the tail, body and head of the arrow. Its contents and meaning 
are described below. 

\item {\bf Arrow description } \type{bend} is a numerical angle to the 
left of the straight line with which the line should bend away from the 
straight line. Negative angles bend to the right. 

\item {\bf MetaPost/MetaFun picture command} (optional) arrow body 
modifiers. These MetaPost/MetaFun command \emph{must} be surrounded by a 
pair of round brackets, \type{"("} and \type{")"}. If you only supply an 
empty pair of round brackets, no commands will be used. 

\item {\bf MetaPost/MetaFun picture command} (optional) arrow tail and/or 
head modifiers. These MetaPost/MetaFun command \emph{must} be surrounded 
by a pair of round brackets, \type{"("} and \type{")"}. If you only supply 
an empty pair of round brackets, no commands will be used. 

\item {\bf Label description} a label string which is typeset by \ConTeXt. 
If the label string is empty, \type{""}, no label will be drawn.

\item {\bf Label description} the linear label position (a numeric value 
between zero and one), which denotes how far along the arrow to place the 
label. Values less than or equal to zero or greater than or equal to one 
will place the label in the \emph{middle} of the arrow. 

\item {\bf Label description} the relative label position (a symbolic 
value) indicates where the label should be placed relative to the position 
along the arrow's length. You can use any value, such as \type{lft}, 
\type{rt}, \type{top}, or \type{bot}, which can be used with the MetaPost 
\type{label} macro. 

\item {\bf MetaPost/MetaFun picture command} (optional) modifies the 
\emph{whole} arrow. In typical MetaPost/MetaFun fashion, these modifiers 
are \emph{outside} of the \type{addArrow}'s formal arguments (that is they 
are everything between the last \type{")"} and the statement's \type{";"}). 

\stopitemize


{\bf The \type{arrowDecorations} string} consists of three parts. The 
first (optional) part describes the tail decorations. The middle 
(optional) part describes the type of arrow. The last (optional) part 
describes the head decorations. An empty, \type{""}, 
\type{arrowDecorations} string will be replaced by the string \type{"->"}. 
The tail and head decorations are deliniated by the \emph{first} body 
decoration\footnote{Any body decorators other than the first one found 
will be ignored.}, \type{"-"}, \type{"="}, \type{"~"}, or \type{"^"}, 
found in the string. If no body decoration is found, the tail decorations 
are assumed to be \type{""} and the body decorations are \type{"-"}. In 
this case all of the \type{arrowDecroations} are used to describe the 
arrow heads. The tail and head decorations are: 

\startitemize

\item \type{"o"} a dot at the tail or head end of the arrow.

\item \type{">"} an arrow head at the tail or head end of the arrow, 
directed in the positive direction of the arrow. 

\item \type{"<"} an arrow head at the tail or head end of the arrow, 
directed in the opposite direction of the arrow.

\item \type{"]"} a second arrow head at the tail or head 
end of the arrow, directed in the positive direction of the arrow. 

\item \type{"["} a second arrow head at the tail or head 
end of the arrow, directed in the opposite direction of the arrow. 

\stopitemize

\noindent\ The body decorations are:

\startitemize

\item \type{"-"} the arrow will be drawn with a simple straight line.

\item \type{"="} the arrow will be drawn with a double \quote{equals} 
line. 

\item \type{"~"} the arrow will be drawn with a squiggly line.

\item \type{"^"} the arrow will be drawn with a zig zag line.

\stopitemize

\section[title=Examples]

\useURL[guideCDPackages][https://www.jmilne.org/not/CDGuide.html]

The best way to understand how these macros can be used is to read through 
the following examples. These examples are similar to those constructed in 
James Milne's \quote{Guide to Commutative Diagram Packages} which can be 
found online at \from[guideCDPackages]. In particular Milne's guide to 
drawing commutative diagrams using the TikZ and/or TikZ-cd packages has 
the greatest number of complex diagrams. We will provide examples of how 
to use \type{t-commdiag} to draw each of these examples. 

\subsection[title=Linear diagrams]

The following diagram shows how to deal with object labels with widly 
different sizes. We have used the optional \type{addObject} text argument 
to modify the location of the two large formulas by shifting them down 5 
big points. 

\startbuffer
\startformula\startMPcode{commDiag}
  setupCommDiags ;
  
  addObject(1,1, "\hat{A}") ;
  addObject(1,2, "\underset{n \in Z}{\prod} A_n") shifted (0, -5bp);
  addObject(1,3, "\underset{n \in Z}{\prod} A_n") shifted (0, -5bp);
  
  drawObjects(1.75cm, 2cm);
  
  addArrow(1,1, 1,2, ">", 0)()()("", 0.5, ) ;
  addArrow(1,2, 1,3, ">", 0)()()("", 0.5, ) ;
\stopMPcode\stopformula
\stopbuffer

\processTEXbuffer

\typebuffer

The following diagram shows how to bend arrows using the \type{addArrow} 
\quote{bend} argument. This example also shows how to use the 
\type{computeArrow} and \type{drawCDArrow} macros\footnote{Documented in 
the Macros chapter below.} provided by the \type{t-commdiag} module. In 
this case the \quote{natural morphism} arrow is not between any pair of 
objects in the diagram, so we have used our own additional 
MetaPost/MetaFun code to draw this morphism. Note that we have used the 
\type{t-commdiag} internal variable, \type{drawFilledArrows}, to draw the 
\quote{natural morphism} with an open arrow head. 

\startbuffer
\startformula\startMPcode{commDiag}
  setupCommDiags ;
  
  addObject(1,1, "A") ;
  addObject(1,2, "B") ;
  
  drawObjects(1.75cm, 2cm);
  
  addArrow(1,1, 1,2, ">",  60)()()("", 0.5, );
  addArrow(1,1, 1,2, ">", -60)()()("", 0.5, );
  
  save upperArrow, lowerArrow, anArrow;
  path upperArrow, lowerArrow, anArrow;
  %
  upperArrow := computeArrow(1,1, 1,2,  50);
  lowerArrow := computeArrow(1,1, 1,2, -50);
  anArrow    := 
    point 0.5*length(upperArrow) of upperArrow --
    point 0.5*length(lowerArrow) of lowerArrow;
  %
  drawFilledArrows := 0;
  drawCDArrow("=>", anArrow)()();
  drawFilledArrows := 1;
\stopMPcode\stopformula
\stopbuffer

\processTEXbuffer

\typebuffer

In the next two diagrams we use the optional \quote{whole arrow modifiers} 
to shift the arrows up or down from the standard arrow positions. 

\startbuffer
\startformula\startMPcode{commDiag}
  setupCommDiags ;
  
  addObject(1,1, "Y \times_X Y") ;
  addObject(1,2, "Y") ;
  addObject(1,3, "X") ;
    
  drawObjects(1.75cm, 2.5cm);
  
  addArrow(1,1, 1,2, ">", 0)()()("p_1", 0.5, top) shifted (0, 3bp);
  addArrow(1,1, 1,2, ">", 0)()()("p_2", 0.5, bot) shifted (0, -3bp);
  
  addArrow(1,2, 1,3, ">", 0)()()("", 0.5, );
\stopMPcode\stopformula
\stopbuffer

\processTEXbuffer

\typebuffer

\blank[2*big]

\startbuffer
\startformula\startMPcode{commDiag}
  setupCommDiags ;
  
  addObject(1,1, "\text{\sans S}(Z)") ;
  addObject(1,2, "\text{\sans S}(X)") ;
  addObject(1,3, "\text{\sans S}(U)") ;
    
  drawObjects(1.75cm, 2.5cm);
  
  addArrow(1,2, 1,1, ">", 0)()()("i^*", 0.5, top) shifted (0, 10bp);
  addArrow(1,1, 1,2, ">", 0)()()("i_*", 0.5,    ) ;
  addArrow(1,2, 1,1, ">", 0)()()("i^!", 0.5, bot) shifted (0, -10bp);
  
  addArrow(1,3, 1,2, ">", 0)()()("i^*", 0.5, top) shifted (0, 10bp);
  addArrow(1,2, 1,3, ">", 0)()()("i_*", 0.5,    ) ;
  addArrow(1,3, 1,2, ">", 0)()()("i^!", 0.5, bot) shifted (0, -10bp);
\stopMPcode\stopformula
\stopbuffer

\processTEXbuffer

\typebuffer

\subsection[title=Rectangular diagrams]

The following diagram shows how to adjust the column width to accomodate 
the \quote{very long label} using the \type{drawObjects} column width 
argument. 

\startbuffer
\startformula\startMPcode{commDiag}
  setupCommDiags ;
  
  addObject(1,1, "A") ;
  addObject(1,2, "B") ;
  addObject(1,3, "C") ;
  addObject(2,1, "D") ;
  addObject(2,2, "E") ;
  addObject(2,3, "F") ;
  
  drawObjects(1.75cm, 4cm);
  
  addArrow(1,1, 1,2, ">", 0)()()("", 0.5, );
  addArrow(1,2, 1,3, ">", 0)()()("\text{very long label}", 0.5, top);
  
  addArrow(2,1, 2,2, ">", 0)()()("", 0.5, );
  addArrow(2,2, 2,3, ">", 0)()()("", 0.5, );
  
  addArrow(1,1, 2,1, ">", 0)()()("", 0.5, );
  addArrow(1,2, 2,2, ">", 0)()()("", 0.5, );
  addArrow(1,3, 2,3, ">", 0)()()("", 0.5, );
\stopMPcode\stopformula
\stopbuffer

\processTEXbuffer

\typebuffer
\subsection[title=Triangular diagrams]

The following diagram shows how to position multiple arrow labels both 
above and below the line of the arrow. This example also shows how to draw 
an arrow with dots by using the (optional) \quote{MetaPost/MetaFun body 
modifiers}. 

\startbuffer
\startformula\startMPcode{commDiag}
  setupCommDiags ;
  
  addObject(1,1, "A") ;
  addObject(1,2, "B") ;
  addObject(1,3, "C") ;
  addObject(2,2, "D") ;
  
  drawObjects(1.75cm, 2cm);
  
  addArrow(1,1, 1,2, ">->", 0)()()("u", 0.5, top);
  addArrow(1,1, 1,2, ">->", 0)()()("b", 0.5, bot);

  addArrow(1,3, 1,2, ">->", 0)()()("u", 0.5, top);
  addArrow(1,3, 1,2, ">->", 0)()()("b", 0.5, bot);
  
  addArrow(1,1, 2,2, "]>",  0)()()("u", 0.5, urt);
  addArrow(1,1, 2,2, "]>",  0)()()("b", 0.5, llft);
  
  addArrow(1,3, 2,2, "]>",  0)()()("u", 0.5, ulft);
  addArrow(1,3, 2,2, "]>",  0)()()("b", 0.5, lrt);

  addArrow(1,2, 2,2, ">",   0)
    (dashed withdots scaled 0.5)()("l", 0.5, lft);
  addArrow(1,2, 2,2, ">",   0)
    (dashed withdots scaled 0.5)()("r", 0.5, rt);
\stopMPcode\stopformula
\stopbuffer

\processTEXbuffer

\typebuffer

\subsection[title=Complex diagrams]

\startbuffer
\startformula\startMPcode{commDiag}
  setupCommDiags ;
  
  addObject(1,2, "A") ;
  
  addObject(2,2, "B") ;
  addObject(2,3, "C") ;
  addObject(2,4, "D") ;
  addObject(2,5, "E") ;
  
  addObject(3,1, "F") ;
  
  drawObjects(1.75cm, 2cm);
  
  addArrow(1,2, 2,3, ">", 0)()()("c", 0.5, );
  addArrow(1,2, 2,4, ">", 0)()()("d", 0.5, );
  addArrow(1,2, 2,5, ">", 0)()()("e", 0.5, );

  addArrow(2,2, 2,3, ">", 0)()()("", 0.5, );
  addArrow(2,3, 2,4, ">", 0)()()("", 0.5, );
  addArrow(2,4, 2,5, ">", 0)()()("", 0.5, );
  
  addArrow(1,2, 3,1, ">", 0)()()("f", 0.5, ulft);
  addArrow(2,2, 3,1, ">", 0)()()("",  0.5, );
\stopMPcode\stopformula
\stopbuffer

\processTEXbuffer

\typebuffer

\blank[2*big]

\startbuffer
\startformula\startMPcode{commDiag}
  setupCommDiags ;
  
  addObject(1,1, "T") ;
  
  addObject(2,2, "X \times_Z Y") ;
  addObject(2,3, "X") ;
  
  addObject(3,2, "Y") ;
  addObject(3,3, "Z") ;
    
  drawObjects(1.75cm, 2.5cm);
  
  addArrow(1,1, 2,3, ">", 30)()()("x", 0.5, urt);
  addArrow(1,1, 2,2, ">", 0)
    (dashed withdots scaled 0.5)()("(x,y)", 0.5, );
  addArrow(1,1, 3,2, ">", -30)()()("y", 0.5, llft);

  addArrow(2,2, 2,3, ">", 0)()()("p", 0.5, top);
  addArrow(2,2, 3,2, ">", 0)()()("q", 0.5, rt);
  addArrow(2,3, 3,3, ">", 0)()()("f", 0.5, rt);
  addArrow(3,2, 3,3, ">", 0)()()("g", 0.5, top);
\stopMPcode\stopformula
\stopbuffer

\processTEXbuffer

\typebuffer

\blank[2*big]

\startbuffer
\startformula\startMPcode{commDiag}
  setupCommDiags ;
  
  addObject(2,1, "0") ;
  addObject(2,2, "\text{Br}(L/K)") ;
  addObject(2,3, "\bigoplus_v \text{Br}(L^v/K_v)") ;
  
  addObject(1,4, "H^2(L/K)") ;
  addObject(3,4, "\mathbb{Q}/\mathbb{Z}") ;
    
  drawObjects(1.75cm, 3cm);
  
  addArrow(2,1, 2,2, ">", 0)()()("", 0.5, );
  addArrow(2,2, 2,3, ">", 0)()()("", 0.5, );
  
  addArrow(2,3, 1,4, ">", 0)()()("", 0.5, );
  addArrow(2,3, 3,4, ">", 0)()()("", 0.5, );
\stopMPcode\stopformula
\stopbuffer

\processTEXbuffer

\typebuffer

\blank[2*big]

\startbuffer
\startformula\startMPcode{commDiag}
  setupCommDiags ;
  
  addObject(1,3, "X \otimes (Y \otimes (Z \otimes T))") ;
  
  addObject(2,1, "X \otimes ((Y \otimes Z) \otimes T)") ;
  addObject(2,5, "(X \otimes Y) \otimes (Z \otimes T)") ;
  
  addObject(3,2, "(X \otimes (Y \otimes Z)) \otimes T") ;
  addObject(3,4, "((X \otimes Y) \otimes Z) \otimes T") ;
    
  drawObjects(1.75cm, 2.5cm);
  
  addArrow(1,3, 2,1, ">", 0)()()("1 \otimes \phi", 0.5, ulft);
  addArrow(1,3, 2,5, ">", 0)()()("\phi", 0.5, urt);
  
  addArrow(2,1, 3,2, ">", 0)()()("\phi", 0.5, llft);
  addArrow(2,5, 3,4, ">", 0)()()("\phi", 0.5, lrt);
  
  addArrow(3,2, 3,4, ">", 0)()()("\phi \otimes 1", 0.5, top);
\stopMPcode\stopformula
\stopbuffer

\processTEXbuffer

\typebuffer

\blank[2*big]

The following diagram shows how to make direct use of the \type{objPos} 
matrix values to construct a path with the correct shape to suit the 
diagram. We then use this path to draw the arrow and its label using the 
\type{drawCDArrow} and \type{drawArrowLabel} macros\footnote{Documented in 
the Macros chapter below.}. 

\startbuffer
\startformula\startMPcode{commDiag}
  setupCommDiags ;
  
  addObject(1,1, "0") ;
  addObject(1,2, "\text{Ker}\; f") ;
  addObject(1,3, "\text{Ker}\: a") ;
  addObject(1,4, "\text{Ker}\: b") ;
  addObject(1,5, "\text{Ker}\: b") ;

  addObject(2,3, "A") withcolor blue;
  addObject(2,4, "B") withcolor blue;
  addObject(2,5, "C") withcolor blue;
  addObject(2,6, "0") withcolor blue;
  
  addObject(3,2, "0")  withcolor blue;
  addObject(3,3, "A'") withcolor blue;
  addObject(3,4, "B'") withcolor blue;
  addObject(3,5, "C'") withcolor blue;

  addObject(4,3, "\text{Coker}\: a") ;
  addObject(4,4, "\text{Coker}\: b") ;
  addObject(4,5, "\text{Coker}\: c") ;
  addObject(4,6, "\text{Coker}\: g'") ;
  addObject(4,7, "0") ;
    
  drawObjects(1.75cm, 2cm);
  
  addArrow(1,1, 1,2, ">", 0)()()("", 0.5, ) withcolor red;
  addArrow(1,2, 1,3, ">", 0)()()("", 0.5, ) withcolor red;
  addArrow(1,3, 1,4, ">", 0)()()("", 0.5, ) withcolor red;
  addArrow(1,4, 1,5, ">", 0)()()("", 0.5, ) withcolor red;
  
  addArrow(1,3, 2,3, ">", 0)()()("", 0.5, );
  addArrow(1,4, 2,4, ">", 0)()()("", 0.5, );
  addArrow(1,5, 2,5, ">", 0)()()("", 0.5, );
  
  addArrow(2,3, 2,4, ">", 0)()()("f", 0.5, top) withcolor blue;
  addArrow(2,4, 2,5, ">", 0)()()("g", 0.5, top) withcolor blue;
  addArrow(2,5, 2,6, ">", 0)()()("", 0.5, ) withcolor blue;

  addArrow(2,3, 3,3, ">", 0)()()("a", 0.5, rt) withcolor blue;
  addArrow(2,4, 3,4, ">", 0)()()("b", 0.5, rt) withcolor blue;
  addArrow(2,5, 3,5, ">", 0)()()("c", 0.5, rt) withcolor blue;

  addArrow(3,2, 3,3, ">", 0)()()("", 0.5, ) withcolor blue;
  addArrow(3,3, 3,4, ">", 0)()()("f'", 0.5, bot) withcolor blue;
  addArrow(3,4, 3,5, ">", 0)()()("g'", 0.5, bot) withcolor blue;

  addArrow(3,3, 4,3, ">", 0)()()("", 0.5, );
  addArrow(3,4, 4,4, ">", 0)()()("", 0.5, );
  addArrow(3,5, 4,5, ">", 0)()()("", 0.5, );
  
  addArrow(4,3, 4,4, ">", 0)()()("", 0.5, ) withcolor red;
  addArrow(4,4, 4,5, ">", 0)()()("", 0.5, ) withcolor red;
  addArrow(4,5, 4,6, ">", 0)()()("", 0.5, ) withcolor red;
  addArrow(4,6, 4,7, ">", 0)()()("", 0.5, ) withcolor red;
 
  save anArrow; path anArrow;
  anArrow := 
    objPos[1][5] shifted (0.75cm, 0) --
    objPos[1][5] shifted (1cm, 0){dir 0} ..
    tension 5 ..
    {dir 0}objPos[4][3] shifted (-1cm, 0) -- 
    objPos[4][3] shifted (-0.75cm, 0);
  drawCDArrow(">", anArrow)()() withcolor red;
  drawArrowLabel("d", 0.51, ulft, anArrow) withcolor red;

\stopMPcode\stopformula
\stopbuffer

\processTEXbuffer

\typebuffer

\stopchapter 